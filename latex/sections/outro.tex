\documentclass[../relazione.tex]{subfiles}
\graphicspath{{\subfix{../images/}}}

\begin{document}
\section{Considerazioni finali}
In definitiva, eseguire algoritmi di scheduling sulla GPU sembra una strada percorribile.

Anche se in questo contesto si è dato per scontato che tutti i task fossero noti all'instante 0, le prestazioni che è in grado di fornire una GPU su grandi quantità di dati da processare restano superiori a quelle delle CPU, soprattutto se si ottimizza il codice per le caratteristiche proprie delle schede video come, ad esempio, la vettorizzazione. Certo è da considerare anche l'overhead che si avrebbe schedulando dei processi tramite GPU che poi andrebbero eseguiti su CPU, tuttavia nel contesto di grossi applicativi o sistemi distribuiti questa può essere una valida alternativa.

Inoltre l'algoritmo implementato in questo esperimento è stato notevolmente rallentato dal fatto che OpenCL non consente la sincronizzazione tra workgroup diversi, cosa che nell'eventualità di un'implementazione reale di scheduling su GPU potrebbe essere evitata lavorando ad un livello più basso di astrazione o, addirittura, sviluppando un kernel apposito per interfacciarsi con la scheda video.
\end{document}